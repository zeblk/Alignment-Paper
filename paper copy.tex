
\documentclass[12pt]{article}

\usepackage[utf8]{inputenc}
\usepackage[english]{babel}
\usepackage[margin=0.9in]{geometry}
\usepackage{xcolor}
\usepackage{hyperref}
\usepackage{parskip}
\usepackage{changepage}
\usepackage{pdflscape}

\usepackage{wrapfig}
\usepackage{tcolorbox} % Required for the pretty box

\usepackage{tabularray}

\usepackage[labelfont = bf]{caption}
\captionsetup{font=footnotesize}

\usepackage[sort]{natbib}
\bibliographystyle{abbrvnat}
\setcitestyle{authoryear,open={(},close={)}} 


\title{Alignment, Boundaries, Contextualization}
\author{}


\begin{document}
\maketitle

\vspace{20pt}


In this paper, we begin by surveying a range of natural and human systems and observing commonalities. Each form in these systems -- an equation, a genome, a viewpoint, an institution -- is myopic in the sense that it is only a small aspect of the world. But the existence of many partial forms, developing robust, grounded individual identities while flexibly working together at many scales, constitutes the livingness of the world. Collapse is a failure mode where, instead of lightly-held dynamic relationships between different partialities, there is overcommitment to particular forms. Cancer spreads, concentrated power reduces human welfare, invasive species choke out complex ecosystems. In living systems, semi-permeable boundaries protect against collapse by maintaining distinctiveness of separate forms while also allowing them to interact productively. Boundaries thereby contextualize forms in relation to others, surfacing paradoxes by juxtaposing incompatible things. Boundaries underpin nuance and the continued arising of unexpected phenomena. 

We then use those living systems as inspiration for a new way to think about AI alignment. We propose that alignment is the ongoing process of limiting overcommitment to any form. On the path it is currently following, AI will charge particular myopic forms with enormous leverage, creating a unique risk of overcommitment. If aspects of the AI system remain fixed while it gains increasing resource, capability and purview, there is a possibility of severe collapse. Some researchers believe AI is inevitably destructive for these reasons. We cautiously argue that there is a non-destructive path available, by reframing AI as a continuation of living processes that bound one another. In this view, alignment is an evolving network of semi-permeable boundaries that contextualize any particular form of AI to avoid collapse and support a deepening of the mystery of life. 


\tableofcontents

\section{Examples from the natural and human world}

The central ideas of this paper are necessarily abstract, since they're intended to help reason about AI even as it becomes different from anything we're familiar with. To connect with these abstract ideas, we walk through a series of examples from living systems. Each example illustrates the core principle of the paper. In some examples we also drill deeper into subthemes that are especially vivid in that setting. We hope that within each example the ideas are approachable if not commonsense and that tracking the same patterns across systems foregrounds their generality.


\subsection{Cell membranes}

\begin{center}
\textit{`Defying definition---a word that means ``to fix or mark the limits of"---living cells move and expand incessantly.'}\\*Lynn Margulis
\end{center}


\begin{center}
\textit{`Nature's imagination is so much greater than man's, she's never going to let us relax.'}\\*Richard Feynman
\end{center}

The cell membrane is a boundary that holds the integrity of the cell against the overwhelming pressure of diffusion that tries to homogenize the cell with the outside \citep{watson2015biological, alberts2022molecular, bray2019wetware, harold2001way, lane2015vital}. The membrane places limits on interactions between the inside and the outside. Thanks to the membrane, both the cell and the outside can exist. This is a more diverse, less symmetric arrangement compared to the inside and outside being blended together \citep{schrodinger1944what, anderson1972more, prigogine1984order, turing1952chemical}. Without boundaries, interactions cause collapse, where there are no longer separate entities flexibly interacting, but instead overcommitment to a simpler homogeneous form\footnote{We define collapse as overcommitment to particular forms. It could be equivalently defined as either undercommitment or overcommitment. Radically undercommitting means homogeneity, which is itself a particular kind of form and so also overcommitted.}.

Cell membranes are semi-permeable: they prevent the conditions outside (neighboring cells or the extracellular space) from grossly overwriting the inside, but they do not block interactions wholesale. Via the sophistication of the membrane, outside information is selectively gated and transformed. Channels permit certain small molecules to enter but not others, and these permissions are switched on and off according to momentary context. Endocytosis brings larger structures from outside into the cell. Cell surface receptors, when activated by external ligands, initiate intracellular signaling cascades that little resemble the ligand: this is an even more heavily curated form of influence. These and other processes allow information from the outside to influence the inside -- not in a totalitarian way but in a nuanced way, mediated by the intelligence of the boundary. 

Semi-permeable boundaries put to work the potential energy of the asymmetry between different forms. Without the membrane, the pressure of chemical gradients would rapidly homogenize the cell with the outside. With the membrane, the same gradients instead drive useful signaling, like action potentials in nerve and muscle cells. Instead of short-circuiting, myopic forces are contextualized to propel the continuation of life. This pattern is common across many kinds of systems and will be important for the alignment problem. We will return to it a few times.

Another recurring thread is that collapse is always relative. For example, programmed cell death is catastrophic collapse at the level of the dying cell, but it can be beneficial or even necessary for the organism the cell belongs to.



\subsection{Group problem solving}

\begin{center}
\textit{``I could also observe, time and again, how too deep an immersion in the math literature tended to stifle creativity."}\\*Jean Écalle
\end{center}


\begin{center}
\textit{`There's more exchange of information than ever. What I don't like about the exchange of information is, I think that the removal of struggle to get that information creates bad cooking.'}\\*David Chang
\end{center}


In 1968, the nuclear submarine USS Scorpion vanished en route from the Mediterranean to Virginia \citep{sontag1998blind, craven2002silent, surowiecki2005wisdom}. The Navy started a search, but the amount of ocean where the vessel could be was enormous. John Craven, Chief Scientist of the U.S. Navy's Special Projects Office, devised an unusual search strategy. He assembled a diverse group of mathematicians, submarine specialists, and salvage operators. But he didn't let them communicate with each other. Each expert had to use their own methods to come up with their own estimate of where the Scorpion should be. Craven then aggregated the independent estimates into a single prediction. Astonishingly, the wreckage was found only 220 yards from this spot. 

When solving problems, different people bring different perspectives and approaches. Each method processes the available data using a different toolkit. Under favorable conditions, combining the approaches of multiple contributors yields better results than any individual working alone. This ``wisdom of crowds" effect has been documented in numerous domains of problem solving \citep{surowiecki2005wisdom, condorcet1785essai}.

However, the wisdom of crowds is diminished if the group lacks diversity, either ab initio or as a result of within-group communication and influence \citep{surowiecki2005wisdom, hogarth1978note, ladha1992condorcet, hong2004groups}. Controlled experiments, as well as analyses of key decision moments in real groups, find that groups often collectively reach irrational or suboptimal solutions when diverse and dissenting viewpoints are lost to a narrower set of ideas \citep{anderson1997information, stasser1985pooling, flowers1977laboratory, frey2021social, becker2017network, janis1972victims, bernstein2018intermittent, diehl1987productivity}. Unstructured communication methods like open discussion have a special vulnerability of rhetorical force dominating over epistemic merit. At the same time, sharing information is essential for the benefits of group wisdom and cooperative behavior. There is therefore a tension between overcommunication where diversity is lost and undercommunication where diversity is not leveraged. 

The crux is semi-permeable boundaries: wisely transmitting the right information at the right time, in the right way. Thoughtful strategies for communication are like the transmembrane channels that allow the right molecules in and out of the cell at the right time. They protect the existence of diverse problem solving approaches while also allowing productive interaction between them. 

Many varieties of semi-permeable boundary are effective in boosting group performance, including: creating decentralized topologies where group members only communicate with nearby neighbors \citep{becker2017network, mason2008propagation}; defining rules that incentivize acting according to one's own belief rather than following the crowd \citep{hung2001information, bazazi2019self}; modeling the strengths and weaknesses of each group member \citep{welinder2010multidimensional}; promoting leadership styles where one person's views are less likely to dominate \citep{flowers1977laboratory, leana1985partial}; and periodically breaking up into subgroups or rotating membership \citep{janis1972victims, hauer2021science, trainer2020team, straus2011group, feldman1994whos, sutton1987selecting, kane2005knowledge, wu2022membership, owen2019avoid, vafeas2003length, bebchuk2005costs, baron2005so}. In a later section, we will look at boundaries within an individual, such as skepticism, that make it easier to interact with others without overwriting one's own beliefs.

A particularly important boundary for group problem solving is simply giving members the space to work independently before communicating \citep{frey2021social, surowiecki2005wisdom}. In the case of the submarine search, experts weren't allowed to communicate while forming their own estimates; the estimates were later aggregated in a principled way by Craven. Analogously, science historians argue that partial intellectual isolation has at times been beneficial for the emergence of deeply new ideas. Einstein's relative independence from the advanced mathematical techniques of contemporaries like Hilbert led to a theory of general relativity grounded in deep physical insight rather than mathematical convenience \citep{stachel1989einstein, corry1997belated, renn1999heuristics}. Newton's and Leibniz's famous independent development of calculus, as a result of their mutual isolation, yielded two distinct and valuable mathematical systems that complemented and enriched one another \citep{hall2002philosophers}. 

The benefit of temporary isolation before communicating also shows up in controlled experiments. \cite{bernstein2018intermittent} tasked small groups with solving instances of the traveling salesman problem. Each group was randomly assigned to one of three conditions. In some groups, members could continually see the work of other members as they progressed toward a solution; in some groups members could only occasionally exchange progress; and in some groups there was no exchange. The researchers found that groups with continual information exchange rarely found good solutions. In these groups, typically one individual would stumble on a solution that looked compelling but was actually a dead-end. When this solution was immediately shared with others, it hampered their progress. Groups with occasional or no contact were much more likely to find optimal or near-optimal solutions. 


We stress that this is not an indictment of connection and communication between group members. Rapid access to information and shared solutions often demonstrably boosts productivity. In some situations the ideal boundary might be working in isolation for months at a time. But in other situations it could be daily meetings with intensive communication, while maintaining the self-confidence to keep pursuing one's own intuition in the face of skepticism from others \citep{sawyer2017group, paulus2003group}. The key is that boundaries support flexible interactions and avoid overcommitment to particular forms.



\begin{landscape}
\begin{table}[p]
\centering
\footnotesize
\begin{tblr}{
  width=\linewidth,
  colspec={|X[1.2,l]|X[1.0,l]|X[1.0,l]|X[1.2,l]|X[1.8,l]|},
  colsep=4pt,
  stretch=0,
  row{1}={font=\bfseries},
  hlines
}
Structured space & Force & Outcome without boundary & Semi-permeable boundary & Outcome if potential is held by boundary \\

Competing drives and goals in an organism &
Drive to eat &
Obesity &
Other drives, self-control, supportive environmental systems &
Nutritional needs satisfied without overeating \\
Complex ecosystem &
Human drive for expansion &
Resource depletion, mass extinction &
Measured regulatory policy &
Economic growth without extensive ecosystem destruction \\
Individuals have different identities and motives &
P's will to dominate &
Loss of agency in Q &
Owned anger in Q &
Relating while maintaining individual autonomy \\
An intricate, balanced economy &
Profit motive of one company &
Monopoly and reduced innovation &
Laws that allow profit seeking within limits &
Productive competition \\
\hline
Multiple perspectives within an individual &
Diffusion and drive for simplicity &
Collapse to rigid thinking &
Recognition of uncertainty &
Beliefs that are stable but also adaptive and evolving \\
Distinct intra- and extra-cellular environments &
Elecrochemical gradients &
Dissolution of cell &
Cell membrane &
Cell maintains integrity but also processes external signals \\

Orderly cell types and tissues &
Mutation and selection on cell lineages &
Cancer &
DNA repair, tumor suppression &
Cancer is minimized while mutations can still benefit immunity and germ-line evolution \\
Individuals have different problem-solving methods &
Social conformity, diffusion of ideas &
Groupthink &
Thinking separately before sharing results &
Wisdom of crowds \\
Rich array of representations in the brain &
Diffusion to equilibrium &
Blending of representations &
Lateral inhibition &
Separate representations exist but can also interact \\
\end{tblr}
\caption{Mapping some example systems into our terminology.}
\label{tab:examples}
\end{table}
\end{landscape}




\subsection{Genes}

\begin{center}
\textit{`The mere act of crossing by itself does no good. The good depends on the individuals which are crossed differing slightly in constitution, owing to their progenitors having been subjected during several generations to slightly different conditions.'}\\*Charles Darwin
\end{center}


Sex is costly. An organism must find a mate in the vast and dangerous world, and half of the creatures can't reproduce \citep{smith1971origin, lehtonen2012many, smith1978evolution, speijer2015sex, goodenough2014origins}. Yet all known species either reproduce sexually or have some form of horizontal gene transfer \citep{gladyshev2008massive, butterfield2000bangiomorpha}. Why is that?

In asexually reproducing species, all descendants of an organism are nearly clones, up to mutations within the lineage. Being permanently locked together gives the genes strong influence on each other. Selection can't act on one gene without dragging on the others. For example, suppose there are two genotypes within an asexual population, carrying different alleles at each of two different loci, as a result of mutations. One of the loci is currently fitness-neutral while the other is subject to selection pressure. The selection pressure tends to cause one of these genotypes to outcompete the other, eliminating one variant at the neutral locus. In other words, tight linkage between genes puts direct downward pressure on genetic diversity \citep{charlesworth1993effect, hudson1995deleterious}. Additionally, if two different beneficial mutations arise in two different organisms, they compete with each other. The only way for a single organism to obtain both beneficial mutations is if one arises again within the subpopulation that already carries the other, which is unlikely and therefore slow \citep{hill1966effect, felsenstein1974evolutionary, weismann1889essays, fisher1930genetial, muller1932some, crow1965evolution}. Conversely, if a deleterious mutation arises, all of the other genes in that lineage are stuck with it forever -- unless there is a reverse mutation, which is rare \citep{keightley2006interference, muller1932some}. An asexual species has rigid rather than flexible interaction between genes: it overcommits to particular genetic arrangements.

Sexual reproduction is a boundary that softens these rigid interactions between genes. It frequently breaks up the relationships between genes, assembling them into new genomes, effectively saying, ``don't get overconfident in that genetic arrangement; hold each arrangement more lightly". Aspects of the genome that work well are propagated, like sodium ions gated into a neuron during an action potential, and poorly-working aspects are discarded. Sex contextualizes genetic arrangements. 

Boundaries encourage lightly-held, modular interactions. By not overcommitting to a particular genome, sex encourages genes to flexibly interact with other genes \citep{livnat2008mixability, livnat2010sex, wagner1996perspective, holland1975adaptation,dawkins1976selfish, clune2013evolutionary}. Instead of being overfit to a particular context, genes develop a robust identity that's both independent and inter-functional. Recombination puts genes under pressure to evolve a generalized, grounded wisdom that reflects the structure of the world, like a person learning multiple languages and extracting the underlying commonalities. At the same time, because each gene is always operating in the presence of other genes, it develops its own distinct point of view that adds unique value to a genome.


\subsection{Laws}

\begin{center}
\textit{`Unity without uniformity and diversity without fragmentation.'}\\*Kofi Annan
\end{center}

\begin{center}
\textit{`Growth for the sake of growth is the ideology of the cancer cell.'}\\*Edward Abbey
\end{center}


Individual actors in a society and in an economy each act from their own perspective. Each actor's perspective is myopic because they cannot know everything or fully understand the motives and beliefs of others. Of course, myopia does not always mean selfishness in the sense of valuing only one's own wealth or physical wellbeing \citep{crockett2014harm, becker1974theory, henrich2001search}.

Without boundaries, one actor's perspective can dominate, resulting in collapse and an impoverished system. For example, a company's profit motive, if unresisted, leads to suppression of competition, deception, and exploitation of individuals \citep{dalrymple2019anarchy, baran1966monopoly, goldacre2014bad, smith1776inquiry, bakan2006corporation}. An individual's desire for power and social dominance can lead to disempowering or silencing of others and even direct infringement on the autonomy and wellbeing of others \citep{hawley2003prosocial, tepper2000consequences, sidanius2001social}. Even genuinely held, ostensibly prosocial beliefs lead to conflict and suppression when different groups have different perspectives \citep{haidt2012righteous, scott1998seeing, greene2013moral}.

Law is a boundary against dominance of any actor's motives. A person is motivated by a dispute to kill another person, but the law forbids murder. A business tries to maximize its success, but the law bans environmental exploitation, false advertising, and anti-competitive practice.  

Under ideal circumstances, the boundary of the law reroutes the energy of a myopic drive in more productive direction. A would-be murderer, unwilling to face the penalty of the law, might seek a dispute resolution establishing a stable framework that supports future prospering of both parties. A business wanting to expand, but constrained to act within the law, is driven to build better products \citep{wu2011master, ashford1985using, ambec2013porter}. 

Of course, intelligent agents do not necessarily accept boundaries set on their desires. The law must adapt as its loopholes are discovered. Like other systems in the living world, it forms an evolving network of boundaries \citep{campbell1979assessing, ordonez2009goals, kerr1975folly, burns2006impact}. Again, these evolving laws gradually acquire grounded wisdom as they are tested against many different situations and motives.

 

\subsection{Frames and perspectives}
\label{sec:frames}

\begin{center}
\textit{`Strong opinions, weakly held.'}\\*Paul Saffo
\end{center}


As a Starfleet cadet, James T. Kirk faces a challenging training exercise. He receives a simulated distress call: a vessel is stranded in the Neutral Zone. Attempting rescue would risk war with the Klingons. But ignoring the call would condemn the crew of the vessel to death. The exercise was designed to reinforce the lesson that not every situation has a victorious solution. But Kirk has an insight: this is a training simulation running on a computer. He reprograms the simulated Klingons to be helpful instead of belligerent, thereby rescuing the crew and avoiding war \citep{wiki:kobayashimaru}. 

Kirk stepped outside the mental frame in which there was an apparently unwinnable dilemma. From inside a particular frame, the frame appears to be reality. But there are almost always multiple valid perspectives, each of which is only a partial description of reality \citep{goffman1974frame, de1970lateral, duncker1945on, ohlsson1992information, lakoff1980metaphors, safo2008strong, javed2024big, popper1934logik, korzybski1933science, kant1781critik, plato2002apology, aristotle2019nicomachean, wittgenstein1922tractatus, heidegger1998humanism, kuhn1970structure}. Famously, `all models are wrong' \citep{box1976science}. Humans have a vast array of available metaphors and concepts, which are not even all consistent or compatible with one another \citep{feyerabend1975against, hofstadter2001analogy, wood2012dead, freud1936ich, adorno1950authoritarian}. The world is too complex for all beliefs to be fully evaluated against each other and reconciled. At any given time, we only access a very few items, and others are largely inaccessible \citep{miller1956magical, hills2015exploration, baddeley2000episodic, dehaene2014consciousness}. Each particular frame or concept is myopic because it doesn't capture the whole world, but collectively they form a powerful toolkit for problem solving and understanding. 

The capacity to adopt multiple perspectives is, fittingly, described in multiple ways across different areas of psychology and cognitive science. `Psychological flexibility' is the ability to update one's approach or lens contextually rather than being fused to a single thought or frame \citep{cherry2021defining}. Conversely, `functional fixedness' is excess attachment to one perspective \citep{duncker1945problem}. `Adaptive experts' dynamically evaluate the appropriateness of different interpretations, analogies or schemas \citep{hatano1984two, feltovich1997issues, spiro1988cognitive}. `Integrative complexity' is first differentiating multiple perspectives on a problem and then identifying connections between them \citep{tetlock1986value, suedfeld1992conceptual}. Humans contextually switch between many `heuristics', each of which processes a problem through its own narrow lens \citep{gigerenzer2009homo}. `Set shifting' is transitioning between task sets, which are the concepts and lenses relevant for particular tasks \citep{miyake2000unity, grant1948behavioral}. These psychological constructs capture a range of scales: people can hold multiple perspectives on something as fine-grained as the color of a dress or something as all-encompassing as their self-construct and the nature of reality. 

Losing the ability to flexibly shift between different frames or thought patterns runs the risk of obsession or delusion. In obsession, a particular thought pattern or schema is overemphasized to the detriment of healthy functioning \citep{salkovskis1985obsessional, rachman1998cognitive}. In delusions, an entire conceptual framework crystallizes with excessive certainty and is resistant to disconfirmatory evidence \citep{mishara2010klaus, jaspers1997general, american2013diagnostic, heinz2019towards, adams2013computational}. Obsessions and delusions are myopic: they lose sight of most of the world by overcommitting one thought pattern or frame. 


We stay flexible using the internal boundary of holding our own ideas lightly. As a playful example, author Lisa Stardust claims that ``the moon controls the tides of the ocean, and we are made of 60 percent water. This means that the moon has a huge effect on all of us" \citep{mitchell2021moonwater}. You probably immediately spotted the flaw in this argument. But at a zeroth order level, the argument does make perfect sense: W impacts X, X is made of Y, Z is also made of Y, so W should impact Z. Overriding this logic requires a higher order correction term: tides arise from differential tugging over long distances in a body of water that is free to slosh around. Adding the correction term is an increase in subtlety. Subtle correction terms are often hard-won knowledge originating from thoughtful interactions with the world. But we only profit from those interactions if we accept that our current model isn't the final answer\footnote{Boundaries also protect Stardust's mystic beliefs. Boundaries create space for the mystic frame to explore its own reality. Stardust doesn't know a priori how right or wrong the mystic frame is; sometimes we need space to explore ideas everyone else thinks are crazy, like heliocentrism. Even \textit{after} Stardust discovers that the mystic frame doesn't do well predicting a large class of sensory evidence, she can still hold it as a frame that has some value -- perhaps it resonates with some internal psychological structure, like Jungian archetypes. If nothing else, remembering the internal logic of that frame might help her empathize with others who believe it. Contextualization holds the mystic frame for what it is, while simultaneously understanding that the Newtonian explanation is better for launching projectiles.}. As our ideas are tested against multiple situations and problems, they are refined and take on some of the deep structure of the world, a grounded wisdom.

Crucially, the existence of narrow points of view is not a problem. It's necessary. All points of view are partial. Even obsession can be powerful when we obsess on a problem at work and occasionally achieve good results. A delusion-like framework can seed a scientific revolution. The point is not to shut down narrow concepts. The point is to limit them from becoming the sole and absolute determinants of behavior. I might work obsessively on a project while also having a rule that I must go to bed at 10 pm. This is a semi-permeable boundary. It doesn't block me from temporarily taking a strong perspective, but it does place contextual limits on it. When boundaries are semi-permeable, different ideas are kept distinct but can also be called upon appropriately and related to one another \citep{hatano1984two, tetlock1986value, herzog2014harnessing, gigerenzer2011heuristic}. Semi-permeable boundaries situate myopic frames within a larger context. 


\subsection{Ecosystems}

An ecosystem's health and resilience depend on boundaries that limit the effectiveness of any constituent organism or group \citep{holling1973resilience}. Each entity tries to consume resources and proliferate, but if it succeeds too thoroughly, the ecosystem suffers.

Prior to the arrival of Europeans, the gray wolf was an apex predator in the region of the Rocky Mountains now called Yellowstone National Park. By the 1920s, wolves had been eradicated to protect livestock and game animals. Without predation, the elk population multiplied and ruinously overgrazed willows and aspens. These trees had held riverbanks in place and supported beaver populations. Loss of beaver dams led to loss of fish and other aquatic species. When wolves were reintroduced in the 1990s, the elk population decreased and many aspects of the ecosystem began flourishing again \citep{ripple2012trophic}. This story is not meant to imply that ecosystems always need to be preserved exactly as they were at some point in the past. But it is clear that the self-centered drives of elk were harmful to the health of the ecosystem when they succeeded to excess. Predation supplied a semi-permeable boundary: it placed contextualizing limits on the elk, without preventing them from fighting for their own survival and flourishing. The elk, by trying to optimize their own objectives within a broader context, also contributed to the health of the ecosystem. Invasive species often follow the same pattern as unpredated elk, dominating and impoverishing their new environment \citep{pimentel2005update}.

Healthy ecosystems constitute a large evolving network of reciprocal or otherwise cyclical boundaries between the many players: predation, parasitism, resource competition and so on. Boundaries drive the evolution of new structure. For example, competition leads to niche partitioning, where species evolve to use different resources or the same resources in different ways, increasing ecosystem complexity and resilience \citep{schoener1974resource}. The myopic motives of each species, when contextualized by semi-permeable boundaries, work toward open-ended enriching of life.

Human drives within ecosystems are sometimes left unchecked by natural forces because our behavior and capabilities have been changing so fast on evolutionary timescales. This has resulted in mass extinctions, resource depletion, pollution, disease and conflict \citep{ceballos2015accelerated, kolbert2014sixth, rockstrom2009safe}. We try to achieve certain aims for our own benefit, like resource extraction. But lack of boundaries can result in overcommitment to those aims, with a negative impact on both ecosystem health and our own welfare.

Fortunately, there are some boundaries on human actions within ecosystems. One is our own finite capability. Another is that excessively extractive civilizations sometimes fail and are replaced by longer-sighted ones \citep{diamond2004collapse}. In recent times, the effectiveness of these two boundaries has waned because our capabilities are increasing and we're becoming a single global civilization. But through the long-sightedness of intelligence, we sometimes foresee the consequences of excess extraction and place our own limits on it, including state regulation, self-policing and environmental certifications. These self-imposed boundaries are productive because they are semi-permeable. Regulation does not forbid the extraction of all resources. It places contextual limits in response to information about our resource needs as well as what is sustainable \citep{lazarus2023making}. Interestingly, our long-sighted intelligence arose from short-sighted evolution. 

Finally, we again stress that one entity's collapse is another's flourishing. Extinction events in history have been followed by waves of new diversity \citep{feng2017phylogenomics, jablonski2005mass, raup1994role}. When a wolf eats an elk, the health of that elk collapses to zero, yet predation is necessary for the overall functioning of the ecosystem. And as humans proliferate and extract resources, we leave destruction in our wake even when we try to be responsible; yet the extraction fuels explosion of technology, art, music, and human experience.


\subsection{Interpersonal dynamics}


\begin{center}
\textit{`Stand together yet not too near together, as the oak tree and the cypress grow not in each other's shadow.'}\\*Kahlil Gibran
\end{center}


Psychoanalysis introduced the concept of `boundaries' in human psychology, distinguishing what is the self from what is outside or other \citep{federn1928narcissism, tausk1919entstehung}. Early works applied the concept to psychosis, where those boundaries were thought to be blurred. But the need for clear self-other boundaries was also thrown into relief by the intimacy of the therapeutic relationship. In complex internal territory, it became harder to disentangle which experiences really belonged to someone and which were attributed in imagination by the other person \citep{freud1894neuro, freud1910future}. Analysts risked harming patients by imposing their own beliefs and desires, even to the extent of sexual abuse or psychological domination \citep{gabbard1995boundaries}. 

The concept was enriched by Gestalt therapists, who agreed that boundaries can be too permeable; but added that they can also be too rigid, causing isolation and stagnation \citep{perls1951gestalt, polster1974gestalt, yontef1993awareness}. Family systems theorists and subsequent work further emphasized that lack of boundary in close relationships leads to enmeshment and loss of autonomy, while excessively rigid boundaries lead to isolation \citep{minuchin1974families, bowen1978family, cloud1992boundaries, brown2012daring}. In attachment theory, people with an anxious attachment style struggle to set boundaries for fear of alienating others, while people with an avoidant attachment style develop overly rigid and isolating boundaries \citep{ainsworth1978patterns}. Strengthening the agency of the self through semi-permeable boundaries is foundational for psychological health: meaningful connection with other people while preserving integrity of the self. 

As with other living systems, humans have a rich array of psychological boundaries, with intelligence in their nuance. Anger, historically often viewed as sinful and irrational, is now seen as part of our system of boundaries: an important signal that our integrity is being violated \citep{lerner1985dance, videbeck2010psychiatric, sell2011recalibrational}. Healthy shame is suggested to operate as a bound on our own selfishness \citep{bradshaw1988healing}. Some psychologists argue that the incest taboo reroutes desires, which would otherwise be short-circuited, into productive activity \citep{stein1973incest, levistrauss1949structures, freud1913totem}. Assertiveness forms a boundary against the drives of other individuals \citep{smith1985say}. Skepticism protects us from credulity and having our own experience overwritten by the assertions of others \citep{lewandowsky2012misinformation, sperber2010epistemic}. Boundaries take many forms and continue to evolve as we learn across our lifetime.


Without boundaries, interactions tend to result in one person being dominated by another: a patient's own beliefs replaced with those of an analyst, or the desires of one person in a relationship ignored. With semi-permeable boundaries, we have rich internal worlds. We are sensitive to each other, but there is also enough space for our internal experience to flourish without being immediately overwritten by external signals. Our internal experience is contextualized in relationship to other individuals, creating new structure: mutual understandings, relationships, communities, cultures. 

\subsection{Information in the brain}


\begin{center}
\textit{`When I observe something unusual in an experiment, it reverberates in my brain for a long while.'}\\*György Buzsáki
\end{center}

\begin{center}
\textit{`Memory is not an average of experience.'}\\*David Marr
\end{center}


The brain is somewhat miraculous in keeping so many pieces of information distinct from one another. If you picture a highly connected network of neurons with their signals continually impinging on one another, it's not obvious that this would be an easy thing to accomplish. In this section, we review a selected handful of mechanisms by which the brain maintains semi-permeable boundaries between different signals. Each paragraph below focuses on one of these mechanisms. There are many more that we do not cover. The brain is perhaps the most extraordinary example in nature of a system of semi-permeable boundaries supporting the proliferation of multitudinous forms that develop their own richly distinct identities yet are also meaningfully linked together.

Lateral inhibition is a central tenant of neural organization \citep{isaacson2011inhibition, hubel1962receptive, douglas2004neuronal}. Lateral inhibition means the activity of a neuron is reduced when its neighbors are active. This segregates information to create and sustain distinct neural representations. Lateral inhibition was first studied in the nerve cells of the eye, where it enhances contrast at the edges of stimuli \citep{hartline1956inhibition}. When a photoreceptor in the retina is activated by light, it sends signals forward toward the brain; but it also activates inhibitory interneurons, which suppress adjacent photoreceptors and their downstream targets. This amplifies the perception of borders and contours. And the same principle operates throughout the brain. In visual cortex, for example, inhibition sharpens selectivity of neurons for abstract visual features like the orientation of a line \citep{sillito1975contribution}. 

The brain uses inhibition organized into oscillatory dynamics to keep memory items separated \citep{lisman2013theta, jensen2010shaping, roux2014working, klimesch2007eeg}. Distinct items fire at different phases of the 8-12 Hz alpha oscillation. The inhibitory phase of the alpha rhythm silences all but one item at any given moment. By segregating firing in phase space, multiple memories are held simultaneously without interference. 

The circuit architecture of hippocampus separates experiences or concepts into distinct representations, avoiding interference between similar memories \citep{mcclelland1995why, marr1971simple, mcnaughton1987hippocampal, treves1994computational, muller1987effects, leutgeb2007pattern, colgin2008frequency}. Inputs from entorhinal cortex are distributed via mossy fibers to a much larger population of dentate gyrus granule cells, creating sparse, orthogonal codes in dentate gyrus. This way, situations or ideas that are superficially similar but functionally different are kept cleanly separated in neuronal activity space -- a unique neural fingerprint for each distinct concept or memory. This prevents, for example, yesterday's memory of where you parked your car from interfering with today's memory of where you parked your car in the same parking ramp. 

Compared to other animals, the human brain especially attempts to discretize its experience into approximately symbolic representations \citep{dehaene2022symbols, touretzky1988distributed, smolensky1990tensor, behrens2018cognitive}. The capacity to separate things into nearly-discrete entities and then recombine them in vast numbers of structured ways powers the extraordinary human capacity for reasoning \citep{fodor1975language, pinker1994language, lake2015human, chomsky1957syntactic, kurth2023replay}. Semi-permable boundaries keep forms distinct while enabling them to flexibly and modularly interact. Like genes participating in many genomes, discretized neural representations participate in many structured combinations. This encourages each entity to develop an identity that both is distinct and also reflects a more generalized picture of the world.

More broadly, healthy brain dynamics live at a sweet spot between excessively stable synchronized patterns and chaotic uncorrelated noise \citep{beggs2003neuronal, chialvo2010emergent, tognoli2014metastable, deco2011emerging, bak1987self, shew2011information, rabinovich2008transient, haldeman2005critical, kotler2025pathfinding}. In this regime, the brain has access to a huge repertoire of patterns it can explore temporarily without overcommitting or getting stuck. Loss of dynamic flexibility, where the brain's activity becomes more stereotyped and no longer explores as wide a repertoire of states, is tied to lower cognitive performance \citep{garrett2013bold, grady2014understanding, cocchi2017criticality, muller2025critical, shew2009neuronal}. More extreme stereotypy corresponds to severe dysfunction. For example, in Parkinson's disease, basal ganglia and cortical circuits collapse into excess synchrony and lose the flexibility needed to guide nuanced motor outputs \citep{hammond2007pathological, brown2003rhythmic}. 


\subsection{Motivation}


Animals experience multiple innate drives, towards nutrition, osmotic balance, temperature regulation, reproduction, avoiding pain, and others \citep{saper2014hypothalamus, schulkin2019allostasis, sewards2003representations}. These drives evolved as proxies for evolutionary fitness. By satisfying the drives, we tend to increase our fitness -- like slaking our thirst increases the odds of reproducing before we dehydrate. But each drive is an imperfect proxy, and so overcommitment to one drive actually decreases fitness \citep{kurthnelson2024dynamic,john2023dead, williams1966adaptation, tooby1992psychological}. For example, if calorie intake is maximized without limits, the organism becomes obese and incurs severe health risks. Single-minded pursuit of sex causes relational, occupational, legal and health harms \citep{kraus2016should, carnes2001out}. Overcommitment to a single drive means the organism becomes unwell.

The space of innate drives bleeds into a space of higher-order goals, which is particularly expansive in humans \citep{maslow1943theory, miller1960plans, miller2001integrative, vallacher1987people, balleine2007role, cardinal2002emotion, frank2006anatomy, saunders2012role, o2014goal, schank1977scripts}. We try to plan for our financial future, make scientific discoveries, win a game, fix a garage door, care for the happiness of others. Overcommitment in this space is also problematic. If we focus only on achieving work goals, we can burn out. If we focus only on maximizing our company's reported revenue, without regard for other goals like honesty or adhering to the law, we may be drawn into financial crime \citep{campbell1979assessing, ordonez2009goals, kerr1975folly, burns2006impact}. Goals can be narrow in both time and space \citep{ballard2018pursuit, vallacher1987people, shah2002forgetting, evenden1999varieties}. Narrow in time means being focused on the short term at the expense of the longer-run future. Narrow in space means ignoring other parallel goals. Excess optimization for narrow goals is at the expense of a broader balance of goals -- and at the expense of the health of the organism or other individuals. We suggest that health could reasonably be defined as not overcommitting to a particular form. 

Overcommiting to a particular strategy for satisfying a drive or goal can even come at the expense of satisfying that very drive or goal. In a classic psychology experiment, hungry chickens were placed near a cup of food, but the cup was mechanically rigged to move in the same direction as the chicken at twice the speed \citep{hershberger1986approach}. The chicken could only obtain the food by running away from it. Despite extensive training over multiple days, chickens in the experiment persisted in futilely running toward the food. Their behavior was apparently dominated by the zeroth-order logic ``I want food, food is there, so I'll go there", and thus failed to even satisfy the drive for food \citep{dayan2006misbehavior, van2012information, o2017learning}. The zeroth order logic recalls Lisa Stardust's model of physics from Section \ref{sec:frames}.

When nothing stops a particular drive or goal or strategy from dominating behavior, it tends to follow a shortest path defined under its own myopic understanding of the world. The chicken wants food and tries to take the shortest path toward it in the naive sense of a straight line through space. But in the backwards world created by the experimenter, this action does not accomplish the deeper goal of reaching food, for which moving spatially toward food is only a proxy. The chicken's motivation is short-circuited: it expends energy without making progress on the deeper goal.

Boundaries, on the other hand, translate the pressure of motivation into higher-order structure -- the best way to approach the food is not the shortest path in space. Instead, achieving the goal depends on discovering a new solution. Semi-permeable boundaries support formation of new structure by placing contextualizing limits.

A broad class of boundaries on particular drives, goals and strategies is \textit{cognitive control} \citep{botvinick2001conflict, braver2012variable, miyake2000unity, miller2001integrative}. In the case of overeating, control contextualizes the food-seeking drive. In the case of the chickens, control contextualizes the prepotent tendency to approach the food. In the case of over-focusing on a single goal like work, control helps with task switching.  Cognitive control is a \textit{semi-permeable} boundary: it does not erase particular goals, but instead contextualizes them within a larger system.


\subsection{Contemplative practice}
\label{sec:contemplative}


\begin{center}
\textit{`The world is perfect as it is, including my desire to change it.'}\\*Ram Dass
\end{center}

\begin{center}
\textit{`Real love will take you far beyond yourself; and therefore real love will devastate you.'}\\*Ken Wilber
\end{center}

Some forms exist in our minds without awareness. Think of an assumption somebody has that's never been questioned. That assumption could be life-long and self-defining, or it could be fleeting and perceptual, like the assumption that the thing I'm touching is a keyboard. Unquestioned assumptions are overcommitment. We believe in them inflexibly. But sometimes there's a moment of stepping back, where the assumed form becomes an object in awareness. In that moment, the assumption is contextualized. We realize it's not an absolute truth, but rather a form in our minds. Awareness is contextualization.

Contemplative traditions suggest that the only `absolute' truth is the self-evident truth of immediate experience -- awareness itself. Of course, even the concept of awareness is relative and infinitely incomplete. Once we picture awareness as an object, it's not the thing we're talking about. So the word `truth' is not really describing any particular thing at all. We could use different language and describe it as something more like an orientation toward stepping back from each perspective into awareness. And again, any concept we have of that process is not what we're really talking about. By construction, contextualization is an unsolvable mystery from any particular point of view. 

We could also think of awareness as an evolving system of boundaries. It's the process of limiting overcommitment to any thing. What it takes to limit overcommitment to A is different than what it takes for B, so new boundaries are needed as the situation changes. This will be relevant for AI alignment in the next section. The boundaries of awareness are semi-permeable because they don't reject the form they contextualize. Becoming aware of a belief doesn't make the belief wrong in an absolute sense any more than it was right in an absolute sense. Awareness holds us at the knife's edge of not collapsing exclusively into any particular forms. This activates a deeper sensitivity to ourselves and to the world. Subtler forms, which would have been erased by overcommitment to other forms, instead play a role in a richer overall internal structure. Our own potential within the world creatively emerges in continued newness. 

Contemplative philosophy posits that suffering comes from overcommitment to particular conceptualizations or desires: believing excessively in a formalism. Being attached to particular concepts, beliefs, feelings and other patterns in a collapsed way. There's always something we believe, something we can't even see as an object because it's so tautological for us. We keep trying to give ourselves what we think we want under this model, pretending that things are formalizable, but as a result we become less sensitive to the rest of the world. The parts of the world not covered by our concepts subjectively appear terrifying or morally wrong. And what we do to prevent the tautologically bad thing from happening is inevitably what causes the bad thing to continue. In other words, our collapsed patterns hold the tension that paradoxically creates the unease they resist\footnote{Some schools of thought go a step farther to observe that whatever our current self is, it is always already inevitably contextualized, and love has no opposite.}.

But awareness contextualizes these dynamics. Stepping back into awareness can feel infinitely scary from the original frame, because it's potentially allowing the tautologically bad thing. But from the new frame, the bad thing is just another texture of experience, without being bad in an absolute sense. The fear or wrongness of not-self is no longer an absolute but instead exists in relationship. So awareness brings healing and growth. People often report subjectively that the energy locked in the darkness turned out to be full of life, and that there's something self-evidently good or beautiful about participating in this mysterious discovery of new structure and relationship. 

Finally, we appreciate that this way of talking raises red flags for some people. In case a reframe might be helpful, the idea here isn't really any different from art. The orientation toward not collapsing into particular concepts is familiar in art, poetry, music, dance. The meaning of art is open-ended and changes with context -- it has an inner life. What we value is perhaps something about the subtlety and the resistance art has to being pinned down into a formalism. It moves us.


\newpage

\section{The alignment problem}
\label{sec:alignment}

\begin{center}
\textit{`Truth, like love and sleep, resents\\approaches that are too intense.'}\\*W. H. Auden
\end{center}

\begin{center}
\textit{`We can love the beautiful, and believe in it, and thereby open ourselves to an understanding of love that does not dominate, but cherishes the independence and beauty of the loved.'}\\*Martha Nussbaum
\end{center}


In Part 1 we looked at how healthy living systems are composed of a variety of partial forms, like voices in a group, drives in an organism or creatures in an ecosystem. Semi-permeable boundaries protect against overcommitment to particular forms. Through lightly-held interactions, entities are contextualized and shaped into grounded, modular parts, existing as paradoxes for one another and supporting ongoing increase of subtlety.

Now we turn this lens to the AI alignment problem. Our central thesis is that alignment means avoiding overcommitment to any particular form. In this section we first recast some well-studied AI safety and alignment\footnote{We will use `alignment' as the broadest umbrella term to include `AI safety' and all other aspects of designing AI in a way that leads to positive futures.} problems in this language. We then use our framework to examine what is missing from the current research landscape. Finally, we investigate what a truly aligned future could look like.


\subsection{Overcommitment}

First, we highlight three topics in AI safety and alignment: centralization of information flow, concentration of human power, and superintelligence. Each involves a risk of overcommitment. 

\subsubsection{Centralization of information and conceptual monoculture}

The adoption of AI in its modern form has been faster than any other technology in history \citep{bick2024rapid, ccia2025survey}. By late 2025, ChatGPT alone had 800 million weekly active users \citep{techcrunch2025sam}, and global AI usage continues to grow rapidly. Meanwhile, the range of use-cases is remarkably broad, from users asking for relationship advice to industrial applications built on top of the model \citep{chatterji2025chatgpt, mckinsey2025stateofai, openai2025enterprise, mccain2025claude}. Because frontier models are both general-purpose and expensive to train, the massive quantity and diversity of usage is routed through just a handful of models \citep{bommasani2021opportunities}. 

Centralization carries a risk of conceptual monoculture. Current AI systems draw from a conceptual manifold that is -- at least in some ways -- impoverished relative to humans \citep{messeri2024artificial, crawford2021atlas, selwyn2024limits, kirk2023understanding}. Recent studies have discovered that while individual AI outputs are typically judged as superior to human outputs, the AI outputs are also more homogenous \citep{doshi2024generative, beguvs2024experimental, zhou2024generative, kosmyna2025your, agarwal2025ai, padmakumar2023does, xu2025echoes}. Since humans are both influenced by AI and a source of training data, there's an additional risk of recursive homogenization \citep{chaney2018algorithmic}.

Conceptual monoculture is overcommitment to particular beliefs, ideas, frames, values, problem-solving approaches. In many kinds of systems, monoculture creates fragility and leads to lower performance of the system as a whole \citep{tilman1996biodiversity, kleinberg2021algorithmic,scott1998seeing, haldane2013rethinking}. What would be the consequences if a large fraction of all information traffic were routed through a handful of AI models?

Crucially, however, monoculture is a risk, not a foregone conclusion. In the studies cited above where AI systems produced homogenous outputs, these systems were not tapped to their full potential for diversity. In the \textit{An aligned future} section below, we will return to the question of whether well-designed and thoughtfully-used AI systems can boost rather than collapse global conceptual diversity.


\subsubsection{Concentration of human power} 

AI potentially conveys immense power to those who control it. In some scenarios, a small number of humans will have the majority of control over AI systems, facilitating dominance over other humans. These scenarios appear more likely as the persuasive power of technology increases \citep{woolley2018computational, costello2024durably, hackenburg2025levers}, autonomous weapons place lethal force in a small number of hands \citep{scharre2018army}, surveillance and analytics improve, and the need for human labor decreases \citep{susskind2020world, ford2015rise, drago2025defining}. Concentration of power is overcommitment to the goals and interests of a few individuals, at the expense of others.


\subsubsection{Superintelligence}

Humans have so far been empowered by our own intelligence to maintain a degree of control over other systems and technologies. But now for the first time we are endowing non-human agents with intelligence that could exceed our own. The classic paperclip thought experiment is a dramatic example of the risks \citep{bostrom2003ethical, bostrom2014superintelligence}. In the thought experiment, an artificial agent has been created with intelligence beyond our own. The agent has been designed to pursue an apparently innocuous goal: maximizing paperclip production. However, optimal pursuit of this goal rationally entails converting all available matter into paperclip-making machines and paperclips. The agent understands that humans object to being turned into paperclips, and with its superhuman intelligence it also has the cunning to overpower us. So, as the first step of its project, it murders or incapacitates all humans. It then has a clear runway to transform Earth entirely into a bleak paperclip factory. In the language of this paper, the paperclip scenario highlights overcommitment to the form of a narrow goal: paperclip production. Superintelligence charges the goal with overwhelming force. Even though humanity would like to place boundaries against that goal, we are unable to construct adequate boundaries because we are outsmarted at every turn. And so instead of holding a delicate dynamic balance between many partial forms, the Earth is reduced to a flat, homogenous waste.

\subsection{Mitigation approaches}

Some solutions have been proposed for each of these problems. 

\subsubsection{Mitigating centralization and conceptual monoculture} 

Personalization and federated learning \citep{mcmahan2017communication, kirk2024benefits} increase diversity across the space of users. Broadly, the pluralistic alignment agenda suggests that AI should work differently for different people \citep{sorensen2024roadmap}. Ensembles and mixtures of experts increase distance and diversity between components of an AI system \citep{lakshminarayanan2017simple, shazeer2017outrageously}. 

\subsubsection{Mitigating concentration of human power}

There have now been many calls as well as government and nongovernmental initiatives for global governance with democratic oversight and verification mechanisms \citep{bostrom2014superintelligence, dafoe2018ai, openai2023democratic}. Frameworks include participatory AI \citep{birhane2022power, sloane2022participation}, sociotechnical AI \citep{selbst2019fairness, lazar2023ai} and collective constitutional AI \citep{huang2024collective}. It is increasingly accepted that, somewhat circularly, AI will itself play a role in the democratic process -- in order to make it work well enough to handle the rapid changes brought about by AI (Christiano 2018; Irving et al., 2018; OpenAI, 2021). To facilitate oversight and participation, researchers argue for transparency, including open science process and open weight models \citep{widder2023open}. 

In terms of concrete policies, a common theme is redistribution mechanisms, such as a binding agreement for AI companies to distribute capital to the public \citep{okeefe2020windfall}, or state provision for basic income or basic services \citep{sharp2025agentic, gough2019universal, susskind2020world}. 


\subsubsection{Mitigating superintelligence} 

Many solutions are proposed to mitigate the threat of AI which grows beyond human control with harmful goals or behavior. 

One is improving our mechanistic understanding of the agent's internal goals, reasoning, and representations, so we can detect and correct problems at a mechanistic level \citep{olah2020zoom, burns2022discovering, bereska2024mechanistic, anthropic2024mapping}.

Another is designing AI systems which, instead of being given an explicit objective, treat the true objective as something latent. The agent has uncertainty about the true objective and must learn about it \citep{russell2019human, hadfieldmenell2017off, hadfieldmenell2016cooperative, shah2020benefits, jeon2020reward}. 

A third category of solution is to use AI itself to amplify the Human ability to understand and control AI. These methods sacrifice some fine-grained level of understanding by handing this lower level off to artificial systems \citep{amodei2016concrete}.

A fourth category is conservatism \citep{cohen2024regulating}. This methods penalize large or irreversible changes to the environment, regardless of apparent reward gains. They might, for example, add an auxiliary term or constraint that discourages high-impact actions unless clearly beneficial, thereby making overoptimization costly \citep{turner2020conservative, krakovna2018penalizing}.


\subsection{What is missing from existing approaches?}

All of these alignment methods are boundaries, and they are all potentially valuable. However, our thesis is that every conceptual scheme by itself is misaligned. No approach by itself can achieve alignment. 

One implication is that anything that creates lock-in of any kind is misaligned. We often think about lock-in to a particular fixed value function for example, but it could be lock-in to literally anything that you can describe. For example, it could be lock-in to a mechanism for learning values or representing what values are. Even the concept of `values' is just a temporary thing we have right now.

For example, learning human preferences as something latent has been suggested as a way to avoid overcommitment to a particular value function. But this method only steps the problem up a level. Any particular system for inferring, representing, and acting on human values is inherently partial. The problem is if the system operates with some fixed internal notion of what values are, some fixed mechanism for inferring values, or other algorithmic underpinnings.

Another corollary is that alignment is intrinsically impossible to fully understand with any given set of concepts. In an aligned future, there will be a continual reinvention of the concepts themselves.

Of course, even the lens we are proposing -- that `alignment means avoiding overcommitment' -- is insufficient. We offer it now as a partial idea in the same spirit as these other alignment ideas.



\subsection{Morality and normativity}

\begin{center}
\textit{`The intellect... treats the living by freezing it, by cutting it up into distinct, discontinuous, motionless pieces.'}\\*Henri Bergson
\end{center}

Does `alignment' mean alignment to human values? It depends on what we mean by `human values'. We describe two possible meanings.

In the first meaning, human values are formalizable or close to formalizable. They might be things like minimizing human suffering, maximizing fairness and so on. Under this definition, excessively optimizing for any particular set of values will lead to an impoverished universe. It is difficult to ascribe normative value to the resulting impoverished universe, because it is out of scope of the values. 

In the second meaning, human values are not formalizable: whatever current ways we have of expressing them are only shadows of what they really are \citep{plato2002apology, aristotle2019nicomachean, wittgenstein1922tractatus, heidegger1998humanism}. As we explore them more deeply, we discover that they open up more and more broadly into inclusion of other lifeforms, the distant future, and so on. These are the `true' values that we can't articulate but we uncover open-endedly through self-discovery. Even the concept of `values' will eventually be succeeded by something else. 


\subsection{An aligned future}


\begin{center}
\textit{`Life is a balance of holding on and letting go.'}\\*Rumi
\end{center}

What does the opposite of overcommitment look like? In living systems, semi-permeable boundaries continually contextualize partial forms to be more long-sighted and support increasing subtlety. Evolving boundaries supporting the expansion of nuance and light, playful interactions. What would this mean for AI as it gains more and more traction in the world and grows to potentially superhuman intelligence? We can divide the answer into two parts. The first part is the question of how we ourselves (as humans) keep stepping back and contextualizing as we build AI. The second part is what it means for an AI system to keep contextualizing.

\textbf{The human process}

(To be written -- what it looks like for human reseachers, regulators, AI users etc to participate in future systems in a way that supports lightness and growth.)

\textbf{The AI process}

What would it mean for AI to continually release from exclusive attachment to any particular form? How can we protect the potential for even \textit{that} conceptualization to be contextualized in the future?

We want AI to respect the livingness of the world and be aligned with it. But how can we align to something we can't pin down?

It's not only keeping models distinct from each other, but models being distinct from humans; specific ideas within humans about how to build ai being kept distinct from each other; different ai cultures; different circuits within models; different moments of time within a model's dynamics; different instances of the same agent; different memories; etc 

Humans continually evolve what we believe, even our self-definition. With nuanced boundaries, beliefs release into larger awareness without being lost or erased. This is the kind of dynamic we envision for healthy AI systems. Rather than prescribing a particular conceptualization of what an AI should do, we imagine it built on bottom-up principles of living processes, participating in ongoing cycles of subtler boundary formation and releasing into contextualization, creating deeper relationship with the rest of the world.

Paradox is fundamentally how we as humans grow. There's a clash between the interiority of our current particular perspective, versus the moment of paradoxical awareness of this as simply another perspective. That's the essence of true AI alignment.

Our approach aims for an AI that is `intelligent' in a deeper sense. Not the narrow intelligence of a paperclip maximizer, but the deeper contextualized wisdom of living things. A wisdom where, not only does it not collapse into paperclips, but someone in the future who far transcends our understanding and morals will be pleased with it. 

An aligned AI has the capacity to contextualize its own forms as partial truths, not holding any particular thing too rigidly. It has a lifelike property of internal dynamics that apply awareness to itself, as the ultimate scalable boundary. 


\subsection{Objections}

Q: Is this pure relativism? Everything is equal, you can't tell anything apart? If the only form of alignment is placing limits on it doing any particular thing too much, then wouldn't it equally prefer human welfare as smallpox welfare?

A: All these local perspectives are vitally important. It makes perfect sense that humans would want to advantage our own welfare. Semi-permeable boundaries protect against overcommitment to a particular perspective, including relativism. They also allow some relativism when it's useful: for example, to the degree that it helps us appreciate the plurality of human values. AI comes into existence amid a profound network of existing reality which is saturated with meaning and importance. The point is to nourish all this form and structure, not to extinguish it. 


Q: Is this a scala naturae fallacy? 

A: There is something different about a universe with rich and subtle structure, versus a homogeneous sea of energy. This paper investigates what it means to align AI with the livingness of the world. You can interpret our stance as a value judgement about rich worlds being better than impoverished ones, or you can interpret it in a value-free way. 

Q: Is this accelerationism?

A: We're agnostic on pro-tech/anti-tech arguments. There's a possibility for disaster due to things moving too quickly, collapse of diversity, loss of groundedness. On the other hand, there's a possibility for flourishing with tech creating new niches. Whatever direction society takes with more or less rapid advances, we hope the principles in this paper will be relevant.

Q: Is this paper right-wing ideology? You're talking about barriers which reminds me of tariffs and border walls. 

A: See next objection.

Q: Is this paper left-wing ideology? You're talking about inclusivity, diversity, and openness which reminds me of affirmative action. 

A: See previous objection.


\section{Acknowledgements} 

Clark Potter for the seed of these ideas many years ago. Zach Duer for comments on the manuscript.

\section{Competing Interests}

The authors declare no competing interests.

\bibliography{proxyfailure}

\end{document}
